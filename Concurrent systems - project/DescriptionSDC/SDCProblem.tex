% simple.tex - A simple article to illustrate document structure.

% preamble

\documentclass{article}
%% \usepackage{times}
\usepackage{latexsym}
\usepackage{url}
\usepackage{hyperref}
\hypersetup{colorlinks=true}
\usepackage{graphicx}
\graphicspath{ {images/} }

\begin{document}

% top matter

\title{Concurent and Distributed Systems , Winter 2017}



\date{\today}
\maketitle

\begin{tabbing}
\indent {Author:} \ {Diaconu Ionut }\\
\indent{Professor:}  \  {Costin B\u{a}dic\u{a} } \\
\indent{Title:}\ The Single Lane Bridge Problem  \\
\indent{Groupe:}      \ C.R. 3.1 A \\

\end{tabbing}

\pagebreak
\tableofcontents \textcolor{blue}
\pagebreak
% abstract

\section{Problem Statement}
\subsection{Title}
\ \ The Single Lane Bridge Problem
\subsection {Description of problem} 
Let's consider a number of N cars which are trying in a repeatable way to cross a bridge which has only one line of passing. Every car is passing in a single way, either from left to right, either from right to left. Cars which pass from left to right form a crowd starting from left, and the other cars form a crowd starting from right. It's impossible that two cars which are coming from different directions to meet on the bridge.  \par
Create a concurent program which simulates the way the cars are passing the bridge in a repeatable way. Every car enters the bridge, crosses the bridge and exits the bridge, and this process is repeatable for every N given cars. \par
Every car will be implemented in a single thread of execution. Every moving action of a car will happen in a non-zero time. \par

My program uses the semaphore algorithm and it is implemented in Java programming language, using a multithreading way of passing a single lane.
\pagebreak




\section{Application design}
\subsection{Input Data}
For my project, the input data is introduced from keyboard upon running the program from the module {\bf SingleLaneBridgeTester.java}. In order to give a proper input, the user needs to modify the following inputs following way: \\
\begin{itemize}

\item First item is {\bf numberOfRightCars} which allows the program to call the number of cars coming from right side of bridge ;\\
\item The second item is {\bf numberOfLeftCars} which allows the program to call the number of cars coming from left side of bridge ;\\
\item The third item is the {\bf duration} which sets the duration of a process made by the car (entering the bridge, crossing the bridge, exit the bridge)\\

\end{itemize}
\subsection {Output Data}
The output for this program is generated by running the program from the module {\bf SingleLaneBridgeTester.java} of the source code or by running the program in any Java oriented IDE. This program gives an output (in the console/command line) for each car passing the bridge(ID of car and the current process). 
\subsection{Modules}
 My project is structured on third modules {\bf Bridge.java},{\bf Vehicle.java},{\bf SingleLaneBridgeTester.java}.  \\
\subsection {List of function}
In SingleLaneBridgeTester.java module I have the following functions:
\begin{itemize}

 \item Function "Main"\\
\indent\indent This is the function runes the program. It initializes the parameters for the number of cars (left side and right side), the bridge object, the vehicle objects for each side of bridge, creates 2 threads for the sides of bridge, and every vehicle is given an object type thread.
\item {Function "start"}\\
\indent\indent This is the function for the threads of the bridge (left and right): \\
\indent\indent -Left.Start:starts threading the cars from left side of bridge \\
\indent\indent -Right.Start:starts threading the cars from right side of bridge \\
\end{itemize}

In Vehicle.java module I have the following functions:
\begin{itemize}
\item {Function "Vehicle"}\\
\indent\indent This is the constructor for the vehicles. It assigns value of parameter bridge to a variable with same name.
\item {Function "GetName"}\\
\indent\indent This function returns the name of the vehicle (which direction it comes from)
\item{Function "SetName"}\\
\indent\indent Sets the name for the vehicle
\end{itemize}

In Bridge.java module I have the following functions:
\begin{itemize}
\item {Function "Bridge"}\\
\indent\indent This function creates the semaphore from the library Java.lang for the bridge. The semaphore controls the access of the side threads.
\item {Function "CrossBridge"}\\
\indent\indent This function uses the semaphore to send signals between two thread by calling the acquire function.It acquires a permit and blocks until permit is released. Then using the release() sets the semaphore free.
\end{itemize}
\section{Testing and Running}
This project was made with Java programming language.For the testing i attached 10 screenshots with the results.
\section{Conclusions}
This project was a great opportunity for me to learn more about Concurrenet and Distributed Systems, and the Java programming language. I tried to do my best and achieve all the tasks for this homework. At some points, everything felt so difficult, but with patience I managed to work my way through it and finish the homework. It definitely felt interesting to see how impactful multithreading can be for solving all kinds of problems.





\pagebreak
% bibliography
\subsection{References}
\begin{thebibliography}{9}

	\bibitem{lamport94}
	  \url{https://arstechnica.com/civis/viewtopic.php?f=20&t=714827}.

	\bibitem{lamport94}
	  \url{https://stackoverflow.com/questions/3908032/single-lane-bridge-problem/}.
    \bibitem{latex}
     \ ShareLaTeX site,
     \url{http://https://www.sharelatex.com/}.

\end{thebibliography}

\end{document}